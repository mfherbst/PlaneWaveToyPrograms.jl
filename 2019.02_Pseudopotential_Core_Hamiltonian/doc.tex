\documentclass{MFHarticle}
\usepackage{MFHcolours}
\usepackage{MFHscience}
\usepackage{todonotes}
\newcommand{\todoil}{\todo[caption={},inline]}

\begin{document}
\title{Expressions for implementing Goedecker, Teter, Hutter,
Hartwigsen pseudopotentials}
\author{mfh}
\maketitle

These derivations follow the papers \citet{Goedecker1996}
and \citet{Hartwigsen1998}.

\section{Terminology}
\begin{description}
	\item[$\vec{K}$] Vector on the reciprocal lattice $\mathcal{R}^\ast$
	\item[$\vec{k}$] Vector inside the first Brillouin zone
	\item[$\vec{q}$] Reciprocal-space vector $\vec{q} = \vec{k} + \vec{K}$
	\item[$\vec{R}$] Vector on the lattice $\mathcal{R}$
	\item[$\vec{r}$] Vector inside the unit cell
	\item[$\vec{x}$] Real-space vector, $\vec{x} = \vec{r} + \vec{R}$
	\item[$\Omega$] Unit cell / unit cell volume
	\item[$e_{\vec{K}}$] Plane wave
		\[ e_{\vec{K}} = \frac1{\sqrt{\Omega}} \exp(\I \vec{K} \cdot \vec{r})\]
	\item[$\Op{T}_{\vec{R}}$] Lattice translation operator
		\[ \Op{T}_{\vec{R}} u(\vec{x}) = u(\vec{x} - \vec{R}) \]
\end{description}

\section{General considerations}
\subsection{Local part}
First we want to compute the matrix elements of the local part
of the pseudopotential.
Given $V_\text{loc}(r)$ as in (1) of \citet{Goedecker1996},
the total local potential is
\[ V_\text{loc}^\text{tot}
= \sum_{\vec{R}\in \mathcal{R}} \Op{T}_{-\vec{R}} V_\text{loc}(r)
= \sum_{\vec{R}\in \mathcal{R}} V_\text{loc}(\norm{\vec{r} + \vec{R}}) \]
assuming there is only one atom per unit cell, which is furthermore
located on the lattice points only.
This potentials necessarily periodic,
thus equal for all of the $\vec{k}$-fibers of the Hamiltonian.
We compute for each $\vec{k}$:
\begin{align*}
	\braket{e_{\vec{K}}}{V_\text{loc}^\text{tot} e_{\vec{K}'}}_\Omega
	&= \frac1{\Omega} \int_\Omega \exp(-\I \vec{K} \cdot \vec{r})
	\left(\sum_{\vec{R}\in \mathcal{R}}
	V_\text{loc}(\norm{\vec{r} + \vec{R}})\right)
	\exp(\I \vec{K}' \cdot \vec{r}) \D\vec{r} \\
	&= \frac1{\Omega} \sum_{\vec{R}\in \mathcal{R}} \left(\int_\Omega
	\exp\big(-\I \vec{K} \cdot \left(\vec{r} + \vec{R}\right) \big)
	V_\text{loc}(\norm{\vec{r} + \vec{R}})
	\exp\big(\I \vec{K}' \cdot \left(\vec{r} + \vec{R} \right)\big) \right)
	\D\vec{r} \\
	&= \frac1{\Omega} \int_{\R^3} \exp\big(-\I \vec{K} \cdot\vec{x}\big)
	V_\text{loc}(\norm{\vec{x}}) \exp\big(\I \vec{K}' \cdot \vec{x}\big) \D\vec{x} \\
	&= \frac1{\Omega} \int_{\R^3} \exp\big(\I (\vec{K}'-\vec{K}) \cdot\vec{x}\big)
	V_\text{loc}(\norm{\vec{x}}) \D\vec{x}\\
	&= \frac1{\Omega} \hat{V}_\text{loc}(\norm{\vec{K}'-\vec{K}})
\end{align*}
\todoil{Can we reiterate on the last equals}
keeping in mind that
\[ \exp\big(\I \vec{K} \cdot \left(\vec{r} + \vec{R}\right)\big) = \exp(\I \vec{K} \cdot \vec{r}) \]

\subsection{Nonlocal part}
First we discuss the properties of the non-local part.
It is effectively composed of projections $\ketbra{p}{p)}$
with $p$ being a projection vector, centred at the atoms.

For simplicity we again assume one atom per unit cell,
located at the lattice points.
Let $p_{\vec{R}}(\vec{r})$ denote the projection vector centred
around the lattice point $\vec{R}$ and let us define
\[ p_{\vec{R}} = \Op{T}_{\vec{R}} p_0,
\qquad \text{i.e.} \quad
p_{\vec{R}}(\vec{x}) = p_0(\vec{x} - \vec{R})   \]
where $p_0$ is the projection vector around the origin.
At first we restrict ourselves to the operator
\[ \Op{M} = \sum_{\vec{R}} \ketbra{p_{\vec{R}}}{p_{\vec{R}} }, \]
which is generated by placing a single projection vector $p_0$
at all lattice points.

We apply $\Op{M}$ to $\Op{T}_{\vec{R}'} u$, where $u$ is arbitrary:
\begin{align*}
\Op{M} \ket{\Op{T}_{\vec{R}'} u}
&= \sum_{\vec{R} \in \mathcal{R}}
\ket{p_{\vec{R}}} \braket{p_{\vec{R}}}{\Op{T}_{\vec{R}'} u}_{\R^3} \\
&= \sum_{\vec{R} \in \mathcal{R}} \ket{p_{\vec{R}}}
\braket{\Op{T}_{\vec{R}} p_0}{\Op{T}_{\vec{R}'} u}_{\R^3} \\
&= \sum_{\vec{R} \in \mathcal{R}} \ket{p_{\vec{R}}}
\braket{\left( \Op{T}_{\vec{R}'} \right)^\dagger \Op{T}_{\vec{R}} p_0}{u}_{\R^3} \\
&= \sum_{\vec{R} \in \mathcal{R}} \ket{p_{\vec{R}}}
\braket{\Op{T}_{\vec{R} - \vec{R}'} p_0}{u}_{\R^3} \\
&= \Op{T}_{\vec{R}'} \sum_{\vec{R} \in \mathcal{R}}
\ket{\left(\Op{T}_{-\vec{R}'} p_{\vec{R}}\right)}
\braket{\Op{T}_{\vec{R} - \vec{R}'} p_0}{u}_{\R^3} \\
&= \Op{T}_{\vec{R}'} \sum_{\vec{R} \in \mathcal{R}}
\ket{p_{\vec{R}- \vec{R}'}} \braket{p_{\vec{R}- \vec{R}'}}{u}_{\R^3} \\
&= \Op{T}_{\vec{R}'} \sum_{\vec{R} \in \mathcal{R}}
\ket{p_{\vec{R}}} \braket{p_{\vec{R}}}{u}_{\R^3} = \Op{T}_{\vec{R}'} \Op{M} u
\end{align*}
where in the last line a change of summation variables occurred.
In other words $\Op{M}$ commutes with $\Op{T}_{\vec{R}'}$
and is thus $\mathcal{R}$-periodic.

Next we compute the fibers of $\Op{M}$
by applying it to a
Bloch wave $\exp\left(\I \vec{k} \cdot \vec{x}\right) u_{\vec{k}}$,
where $u_{\vec{k}}$ is $\mathcal{R}$-periodic:
\begin{align*}
\Op{M} \ket{\exp\left(\I \vec{k} \cdot \vec{x}\right) u_{\vec{k}}}
&= \sum_{\vec{R} \in \mathcal{R}}
\ket{p_{\vec{R}}} \braket{\Op{T}_{\vec{R}} p_0}
{\exp\left(\I \vec{k} \cdot \vec{x}\right) u_{\vec{k}}(\vec{x})}_{\R^3} \\
&= \sum_{\vec{R} \in \mathcal{R}}
\ket{p_{\vec{R}}} \braket{p_0}
{\left(\Op{T}_{-\vec{R}}
	\exp\left(\I \vec{k} \cdot \vec{x}\right) u_{\vec{k}}
\right)}_{\R^3} \\
&= \sum_{\vec{R} \in \mathcal{R}}
\ket{p_{\vec{R}}} \braket{p_0}
{
\exp\left(\I \vec{k} \cdot \vec{R}\right)
\exp\left(\I \vec{k} \cdot \vec{x}\right) u_{\vec{k}}
}_{\R^3} \\
&= 
\sum_{\vec{R} \in \mathcal{R}}
\ket{p_{\vec{R}} \exp\left(\I \vec{k} \cdot \vec{R}\right)}
\ 
\braket{p_0 \exp\left(-\I \vec{k} \cdot \vec{x}\right)}
{u_{\vec{k}}}_{\R^3} \\
&= 
\sum_{\vec{R} \in \mathcal{R}}
\ket{p_{\vec{R}} \exp\big(-\I \vec{k} \cdot \left(\vec{x}-\vec{R}\right)\big)}
\ 
\braket{p_0 \exp\left(-\I \vec{k} \cdot \vec{x}\right)}
{u_{\vec{k}}}_{\R^3} \exp\left(\I \vec{k} \cdot \vec{x}\right)\\
&= 
\sum_{\vec{R} \in \mathcal{R}} \left(\Op{T}_{\vec{R}}
\ket{p_0 \exp\left(-\I \vec{k} \cdot \vec{x}\right)}\right)
\braket{p_0 \exp\left(-\I \vec{k} \cdot \vec{x}\right)}
{u_{\vec{k}}}_{\R^3} \exp\left(\I \vec{k} \cdot \vec{x}\right).
\end{align*}
Since
\begin{align*}
	\braket{p_0 \exp\left(-\I \vec{k} \cdot \vec{x}\right)}
	{u_{\vec{k}}}_{\R^3}
	&= \int_{\R^3} p_0^\ast(\vec{x}) \exp\left(\I \vec{k} \cdot \vec{x}\right)
		u_{\vec{k}}(\vec{x}) \D \vec{x} \\
	&= \sum_{\vec{R} \in \mathcal{R}} \int_\Omega
		p_0^\ast(\vec{r} + \vec{R})
		\exp\left(\I \vec{k} \cdot \vec{r}\right)
		\exp\left(\I \vec{k} \cdot \vec{R}\right)
		u_{\vec{k}}(\vec{r} + \vec{R}) \D \vec{r} \\
	&= \sum_{\vec{R} \in \mathcal{R}} \int_\Omega
		p_0^\ast(\vec{r} + \vec{R})
		\exp\left(\I \vec{k} \cdot \vec{r}\right)
		\exp\left(\I \vec{k} \cdot \vec{R}\right)
		u_{\vec{k}}(\vec{r}) \D \vec{r} \\
	&= \sum_{\vec{R} \in \mathcal{R}} \int_\Omega
		\left(
		\Op{T}_{-\vec{R}}
		p_0^\ast(\vec{r})
		\exp\left(\I \vec{k} \cdot \vec{r}\right)
		\right)
		u_{\vec{k}}(\vec{r}) \D \vec{r} \\
	&= \sum_{\vec{R} \in \mathcal{R}}
		\braket{\left(\Op{T}_{\vec{R}} \ 
		p_0 \exp\left(-\I \vec{k} \cdot \vec{r}\right)\right)}
		{u_{\vec{k}}}_\Omega
\end{align*}
the $\vec{k}$-fibers of $\Op{M}$ are given by $\Op{M}_{\vec{k}}$ with
\[\Op{M}_{\vec{k}} u =
\ket{
\sum_{\vec{R} \in \mathcal{R}}
\left(
\Op{T}_{\vec{R}}\
p_0 \exp\left(-\I \vec{k} \cdot \vec{r}\right)\right)}
\braket{
\sum_{\vec{R}' \in \mathcal{R}}
\left(
\Op{T}_{\vec{R}'}\
p_0 \exp\left(-\I \vec{k} \cdot \vec{r}\right)\right)}
{u}_\Omega,
\]
where $u$ is an $\mathcal{R}$-periodic function
and integration is to be conducted over the unit cell $\Omega$.
The matrix elements of the fiber $\Op{M}_{\vec{k}}$
between plane waves is thus:
\[
\braket{e_{\vec{K}}}{\Op{M}_{\vec{k}} e_{\vec{K}'}}_\Omega
=
\braket{e_{\vec{K}}}{
\sum_{\vec{R} \in \mathcal{R}}
\left(
\Op{T}_{\vec{R}}\
p_0 \exp\left(-\I \vec{k} \cdot \vec{r}\right)\right)}
\braket{
\sum_{\vec{R}' \in \mathcal{R}}
\left(
\Op{T}_{\vec{R}'}\
p_0 \exp\left(-\I \vec{k} \cdot \vec{r}\right)\right)}
{e_{\vec{K}'}}
\]
To compute we consider for a fixed $\vec{k}$ the integral
\begin{align*}
\braket{e_{\vec{K}}}{
\sum_{\vec{R} \in \mathcal{R}}
\left(
\Op{T}_{\vec{R}}\
p_0 \exp\left(-\I \vec{k} \cdot \vec{r}\right)\right)}
&=
\sum_{\vec{R} \in \mathcal{R}}
\braket{e_{\vec{K}}}{p_{\vec{R}} \exp\left(-\I \vec{k} \cdot \vec{r}\right)
\exp\left(\I \vec{k} \cdot \vec{R}\right)} \\
&=
\sum_{\vec{R} \in \mathcal{R}}
\braket{e_{\vec{K} + \vec{k}}}{p_{\vec{R}}
\exp\left(\I \vec{k} \cdot \vec{R}\right)} \\
&= \frac1{\sqrt{\Omega}}
\sum_{\vec{R} \in \mathcal{R}}
\int_\Omega
\exp\left(-\I \left( \vec{K} + \vec{k} \right)
\cdot \left(\vec{r} - \vec{R}\right)\right)
p_{\vec{R}}(\vec{r}) \D\vec{r} \\
&= \frac1{\sqrt{\Omega}}
\sum_{\vec{R} \in \mathcal{R}}
\int_\Omega
\exp\left(-\I \left( \vec{K} + \vec{k} \right)
\cdot \left(\vec{r} - \vec{R}\right)\right)
p_0(\vec{r} - \vec{R}) \D\vec{r} \\
&= \frac1{\sqrt{\Omega}}
\int_{\R^3}
\exp\left(-\I \left( \vec{K} + \vec{k} \right)
\cdot \vec{x}\right)
p_0(\vec{x}) \D\vec{r} \\
&= \frac1{\sqrt{\Omega}} \hat{p}_0(\vec{K} + \vec{k}).
\end{align*}
Therefore
\[
\braket{e_{\vec{K}}}{\Op{M}_{\vec{k}} e_{\vec{K}'}}_\Omega
=\frac1{\Omega} \hat{p}_0(\vec{K} + \vec{k}) \hat{p}_0^\ast(\vec{K}' + \vec{k}).
\]

The actual non-local part of the pseudopotential
$\Op{V}_\text{nloc}$ used in \cite{Goedecker1996,Hartwigsen1998}
is a linear combination of the form
\[
	\Op{V}_\text{nloc} =
	\sum_{\vec{R} \in \mathcal{R}}
	\sum_{l}
	\sum_{i,j}\
	\ket{\Op{T}_{\vec{R}}\ p_i^l} h_{ij}^l \bra{\Op{T}_{\vec{R}}\ p_j^l}
\]
where $l$ runs over all angular momentum channels and
for each $l$ the $h_{ij}^l$
are the elements of an Hermitian matrix.
The $p_i^l(\vec{r})$ are the projection vectors
centred around the atom of the unit cell%
\footnote{Recall we assume 1 atom per unit cell for now.}.
The aforementioned analysis continues through to such a linear combination,
such that its fibers have matrix elements
\[
	\frac1{\Omega}
	\sum_{\vec{R} \in \mathcal{R}}
	\sum_{l}
	\sum_{i,j}\
	\hat{p}_i^l(\vec{K} + \vec{k})
	h_{ij}^l
	\hat{p}_j^{l\ast}(\vec{K}' + \vec{k})
\]
in a plane wave basis.

\section[
Confirming the GTH and HGH expressions
]{Confirming the GTH and HGH expressions \cite{Goedecker1996,Hartwigsen1998}}




\bibliographystyle{mfh}
\bibliography{literatur}
\end{document}
