\documentclass{MFHarticle}
\usepackage{MFHcolours}
\usepackage{MFHscience}
\usepackage{todonotes}
\newcommand{\todoil}{\todo[caption={},inline]}

\begin{document}
\title{Computing the total energy in a Goedecker, Teter, Hutter,
Hartwigsen pseudopotential model}
\author{Michael F. Herbst}
\maketitle

\section{Terminology}
\begin{description}
	\item[$\vec{K}$] Vector on the reciprocal lattice $\mathcal{R}^\ast$
	\item[$\vec{k}$] Vector inside the first Brillouin zone $\mathcal{B}$
	\item[$\vec{q}$] Reciprocal-space vector $\vec{q} = \vec{k} + \vec{K}$
	\item[$\vec{R}$] Vector on the lattice $\mathcal{R}$
	\item[$\vec{r}$] Vector inside the unit cell
	\item[$\vec{x}$] Real-space vector, $\vec{x} = \vec{r} + \vec{R}$
	\item[$\Omega$] Unit cell / unit cell volume
	\item[$\Gamma$] Unit cell
	\item[$e_{\vec{K}}$] Plane wave
		\[ e_{\vec{K}}(\vec{r}) = \frac1{\sqrt{\Omega}} \exp(\I \vec{K} \cdot \vec{r})\]
	\item[$\Op{T}_{\vec{R}}$] Lattice translation operator
		\[ \Op{T}_{\vec{R}} u(\vec{x}) = u(\vec{x} - \vec{R}) \]
\end{description}



\section{Electrostatic energy}

Follow \citet{Payne1992}

\subsection{Smooth, periodic densities}

\todoil{Discuss exact conditions}
Given two compact and smooth charge densities
$\rho$ and $\tilde{\rho}$,
their classical Coulombic interaction is given by the integral
\begin{equation}
	\frac12
	\int_{\R^3 \times \R^3}
	\frac{\rho(\vec{x}) \tilde{\rho}(\vec{x}')}{\norm{\vec{x} - \vec{x}'}}
	\D\vec{x} \D\vec{x}'.
	\label{eqn:ElectrostaticEnergySmooth}
\end{equation}
A special case of this expression is the Coulombic interaction energy,
effectively amounting to $\rho = \tilde{\rho}$ in the expression above.

For the case of a periodic system the charge density $\rho(\vec{x})$ is not
compact, such that integral \eqref{eqn:ElectrostaticEnergySmooth}
is no longer meaningful.
Instead one therefore wishes to look
at the equivalent quantities per unit cell
\begin{equation}
	D(\rho, \tilde{\rho}) \equiv
	\frac12
	\int_{\Gamma \times \R^3}
	\frac{\rho(\vec{x}) \tilde{\rho}(\vec{x}')}{\norm{\vec{x} - \vec{x}'}}
	\D\vec{x} \D\vec{x}'
	\label{eqn:ElectrostaticEnergyPeriodic}
\end{equation}
and
\[
	E(\rho) = D_\Gamma(\rho, \rho).
\]
Unfortunately even for smooth periodic densities $\rho(\vec{x})$
\eqref{eqn:ElectrostaticEnergyPeriodic}
in general does diverge and is thus formally not defined.
To see this consider the Hartree potential term $V_\rho$,
i.e.~the integral
\begin{equation}
	\int_{\R^3}
	\frac{\rho(\vec{x}')}{\norm{\vec{r} - \vec{x}'}}
	\D\vec{x}'.
	\label{eqn:ElectrostaticPotential}
\end{equation}
Already for simple densities, e.g.~a uniform constant density
with $\rho(\vec{x}) = \epsilon > 0$,
the integrand is not even absolutely integrable,
i.e.~the integral
\begin{equation}
	\int_{\R^3} \abs{  \frac{\rho(\vec{x}')}{\norm{\vec{r} - \vec{x}'}}  }
	\D\vec{x}'
	\label{eqn:ElectrostaticPotentialAbsolute}
\end{equation}
diverges.
Thus the integral in \eqref{eqn:ElectrostaticPotential}
does not exist in the Lebesgue sense.
Even if additionally charge neutrality,
\[\int_{\R^3} \rho(\vec{x}) \D\vec{x} =0, \]
is demanded for the density,
the $1/x$ kernel decays too slowly to ensure existence
of the absolute integral \eqref{eqn:ElectrostaticPotentialAbsolute}.

As a remedy one may regularise, i.e. replace the $1/x$ interaction
Kernel by a Yukawa-type interaction
\[ K_\alpha(x) = e^{-\alpha x} / x \]
with $\alpha > 0$ small and consider integrals
\begin{equation}
	D_\alpha(\rho, \tilde{\rho}) \equiv
	\frac12
	\int_{\Gamma \times \R^3}
	\rho(\vec{x})\, \tilde{\rho}(\vec{x}')
	\,
	K_\alpha\!\left(\norm{\vec{x} - \vec{x}'}\right)
	\D\vec{x} \D\vec{x}'
	\label{eqn:ElectrostaticIntegralYukawa}
\end{equation}
and corresponding energies
\begin{equation}
	E_\alpha(\rho) = D_\alpha(\rho, \rho).
	\label{eqn:ElectrostaticEnergyYukawa}
\end{equation}
The integral
\eqref{eqn:ElectrostaticIntegralYukawa} is summable
for a smooth density $\rho(\vec{x})$,
i.e.~a density, which is not a sum of Diracs.
Since $\rho$ is $\mathcal{R}$-periodic one may expand it exactly
in a countably infinite number of plane waves
\[
	\rho(\vec{x}) = \sum_{\vec{G} \in \mathcal{R}^\ast}
		\hat{\rho}(\vec{G}) e_{\vec{G}}
		\qquad \text{and} \qquad
	\tilde{\rho}(\vec{x}) = \sum_{\vec{G} \in \mathcal{R}^\ast}
		\hat{\tilde{\rho}}(\vec{G}) e_{\vec{G}}
\]
and use this to compute the convolution \eqref{eqn:ElectrostaticIntegralYukawa}
in Fourier space as
\begin{align}
	\nonumber
	D_\alpha(\rho, \tilde{\rho}) &=
	\frac12
	\sum_{\vec{G} \in \mathcal{R}^\ast}
	\sum_{\vec{G}' \in \mathcal{R}^\ast}
	\int_{\Gamma \times \R^3}
	\hat{\rho}(\vec{G}) e_{\vec{G}}(\vec{r})
	\,
	\hat{\tilde{\rho}}(\vec{G}') e_{\vec{G}'}(\vec{x}')
	\,K_\alpha\!\left(\norm{\vec{r} - \vec{x}'}\right)
	\D\vec{r} \D\vec{x}' \\
	&=
	\nonumber
	\frac12
	\sum_{\vec{G} \in \mathcal{R}^\ast}
	\sum_{\vec{G}' \in \mathcal{R}^\ast}
	\int_{\Gamma}
	\hat{\rho}(\vec{G}) e_{\vec{G}}(\vec{x})
	\,
	\hat{K}_\alpha(\vec{G}') \hat{\tilde{\rho}}(\vec{G}') e_{\vec{G}'}(\vec{x})
	\D\vec{r} \\
	&=
	\nonumber
	\frac12
	\sum_{\vec{G} \in \mathcal{R}^\ast}
	\sum_{\vec{G}' \in \mathcal{R}^\ast}
	\hat{\rho}(\vec{G})
	\hat{K}_\alpha(\vec{G}') \hat{\tilde{\rho}}(\vec{G}')
	\, \frac{1}{\Omega} \int_{\Gamma}
	\exp\big(\I\, \vec{r} \cdot \left( \vec{G} + \vec{G}' \right)\big)
	\D\vec{r} \\
	&?=
	\nonumber
	\frac12
	\sum_{\vec{G} \in \mathcal{R}^\ast}
	\hat{\rho}(\vec{G})
	\hat{K}_\alpha(-\vec{G}) \hat{\tilde{\rho}}(-\vec{G})\\
	\label{eqn:ElectrostaticEnergyYukawaSeries}
	&=
	\frac12
	\sum_{\vec{G} \in \mathcal{R}^\ast}
	\frac{4\pi \hat{\rho}(\vec{G}) \left(\hat{\tilde{\rho}}(\vec{G})\right)^\ast}
	{G^2 + \alpha^2},
\end{align}
\todoil{One line not clear above.}
where in the last line we used that the density $\rho$ is a real quantity.

For the existence of the limit%
\footnote{Note, that we always need $\rho$ to be smooth here.
	Otherwise the Fourier coefficients $\hat{\rho}$ do not decay rapidly enough
	for the series in \eqref{eqn:ElectrostaticEnergyYukawaSeries}
	to be summable.}
\[ D(\rho, \tilde{\rho}) = \lim_{\alpha \to 0} D_\alpha(\rho, \tilde{\rho}) \]
we are interested in, the DC ($G = 0$) Fourier coefficient has to vanish,
which is exactly the case iff $\rho(G) =0$ or $\hat{\rho}(G) = 0$.
Thus at least one of the densities needs to have a zero charge average,
i.e.~needs to be charge-neutral.

For smooth, charge-neutral densities one may thus compute
the electrostatic energy by a \textbf{summation in Fourier space}.

\subsection{Treating the nuclei-nuclei term}
In a DFT treatment we are interested in the total electrostatic energy
per unit cell, i.e.~the energy of the Coulombic interaction of both
electrons and nuclei combined.
Let
\[
	\rho_N(\vec{x}) = \sum_{\vec{R}\in \mathcal{R}} \Op{T}_{\vec{R}}
	\sum_i Z_i \delta(\vec{x} - \vec{t}_i)
\]
be the periodic and discrete density of the nuclei
and $\rho_e$ the corresponding electronic charge to ensure net
neutrality
\[\int_\Gamma \rho_e + \rho_N \D\vec{r} = 0. \]
Our aim is thus to compute%
\footnote{Note that the sum for the nuclei-nuclei interaction
cannot be replaced
by the simpler integral
$\frac12 \int_{\Gamma \times \R^3} \rho_N(r) \rho_N(x) K(x-r)\D \vec{r} \D \vec{x}$
because such an integral is always ill-defined due to the
contact term at $K(0) = -\infty$.}
\begin{align}
	\nonumber
	\int_{\Gamma \times \R^3}
	&\frac12\rho_e(\vec{r}) \rho_e(\vec{x}') \, K\!\left(\norm{\vec{r}-\vec{x}'}\right)
	+\rho_e(\vec{r}) \rho_N(\vec{x}') \, K\!\left(\norm{\vec{r}-\vec{x}'}\right) \\
	&+
	\sum_i
	\sum_{j>i} Z_i Z_j
	\left(
	\sum_{\vec{R}'\in \mathcal{R}}
	\Op{T}_{\vec{R}'}
	\delta(\vec{r} - \vec{t}_i)
	\right)
	\left(
	\sum_{\vec{R}\in \mathcal{R}}
	\Op{T}_{\vec{R}}
	\delta(\vec{x}' - \vec{t}_j)
	\right)
	K\!\left(\norm{\vec{r}-\vec{x}'}\right)
	\D\vec{r} \D\vec{x}'.
	\label{eqn:ElectrostaticDFT}
\end{align}
where $K(x) = 1/x$ is the Coulomb kernel.

Our aim for computing \eqref{eqn:ElectrostaticDFT}
is to follow the route sketched in the previous section,
i.e.~to replace $K$ by the regularised $K_\alpha$,
express the integral as a Fourier series and take
the limit $\alpha\to0$.
Unfortunately the total density $\rho \equiv \rho_e + \rho_N$
is no longer smooth in this case,
which causes trouble especially in
the last term of \eqref{eqn:ElectrostaticDFT},
where the Fourier coefficients will converge too slowly.

\newcommand{\tlr}[1]{\textcolor{red}{#1}}
\newcommand{\tsr}[1]{\textcolor{blue}{#1}}
To make progress we will smear out the discrete nuclear charges
by a Gaussian blur and carry the error in a second term.
For a given parameter $\eta > 0$ to be discussed later, we set
\begin{align}
	\label{eqn:TermSplitup}
	\nonumber
	\delta(\vec{x}) &=
	\tlr{\left( \frac{\eta}{\pi} \right)^{3/2} \exp\left( -\eta^2 x^2 \right)}
		+ \tsr{\left(
		\delta(\vec{x}) -
	\left( \frac{\eta^2}{\pi} \right)^{3/2} \exp\left( -\eta^2 x^2 \right)
	\right)} \\
	&\equiv
	\tlr{G_\eta(\vec{x})}
	+ \tsr{\left( \delta(\vec{x}) - G_\eta(\vec{x}) \right)}.
\end{align}
Replacing $1/x$ by the Yukawa kernel $K_\alpha(x)$
makes each term of \eqref{eqn:ElectrostaticDFT} convergent.
Therefore we can treat each summand individually.
For example for $i\neq j$ one of the last set of terms
can be rewritten as follows:
\bibliographystyle{mfh}
\bibliography{literatur}
\end{document}
